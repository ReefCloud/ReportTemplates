% Options for packages loaded elsewhere
\PassOptionsToPackage{unicode}{hyperref}
\PassOptionsToPackage{hyphens}{url}
\PassOptionsToPackage{dvipsnames,svgnames,x11names}{xcolor}
%
\documentclass[
  letterpaper,
  DIV=11,
  numbers=noendperiod]{scrartcl}

\usepackage{amsmath,amssymb}
\usepackage{iftex}
\ifPDFTeX
  \usepackage[T1]{fontenc}
  \usepackage[utf8]{inputenc}
  \usepackage{textcomp} % provide euro and other symbols
\else % if luatex or xetex
  \usepackage{unicode-math}
  \defaultfontfeatures{Scale=MatchLowercase}
  \defaultfontfeatures[\rmfamily]{Ligatures=TeX,Scale=1}
\fi
\usepackage{lmodern}
\ifPDFTeX\else  
    % xetex/luatex font selection
    \setmainfont[]{Arial}
\fi
% Use upquote if available, for straight quotes in verbatim environments
\IfFileExists{upquote.sty}{\usepackage{upquote}}{}
\IfFileExists{microtype.sty}{% use microtype if available
  \usepackage[]{microtype}
  \UseMicrotypeSet[protrusion]{basicmath} % disable protrusion for tt fonts
}{}
\makeatletter
\@ifundefined{KOMAClassName}{% if non-KOMA class
  \IfFileExists{parskip.sty}{%
    \usepackage{parskip}
  }{% else
    \setlength{\parindent}{0pt}
    \setlength{\parskip}{6pt plus 2pt minus 1pt}}
}{% if KOMA class
  \KOMAoptions{parskip=half}}
\makeatother
\usepackage{xcolor}
\setlength{\emergencystretch}{3em} % prevent overfull lines
\setcounter{secnumdepth}{-\maxdimen} % remove section numbering
% Make \paragraph and \subparagraph free-standing
\makeatletter
\ifx\paragraph\undefined\else
  \let\oldparagraph\paragraph
  \renewcommand{\paragraph}{
    \@ifstar
      \xxxParagraphStar
      \xxxParagraphNoStar
  }
  \newcommand{\xxxParagraphStar}[1]{\oldparagraph*{#1}\mbox{}}
  \newcommand{\xxxParagraphNoStar}[1]{\oldparagraph{#1}\mbox{}}
\fi
\ifx\subparagraph\undefined\else
  \let\oldsubparagraph\subparagraph
  \renewcommand{\subparagraph}{
    \@ifstar
      \xxxSubParagraphStar
      \xxxSubParagraphNoStar
  }
  \newcommand{\xxxSubParagraphStar}[1]{\oldsubparagraph*{#1}\mbox{}}
  \newcommand{\xxxSubParagraphNoStar}[1]{\oldsubparagraph{#1}\mbox{}}
\fi
\makeatother


\providecommand{\tightlist}{%
  \setlength{\itemsep}{0pt}\setlength{\parskip}{0pt}}\usepackage{longtable,booktabs,array}
\usepackage{calc} % for calculating minipage widths
% Correct order of tables after \paragraph or \subparagraph
\usepackage{etoolbox}
\makeatletter
\patchcmd\longtable{\par}{\if@noskipsec\mbox{}\fi\par}{}{}
\makeatother
% Allow footnotes in longtable head/foot
\IfFileExists{footnotehyper.sty}{\usepackage{footnotehyper}}{\usepackage{footnote}}
\makesavenoteenv{longtable}
\usepackage{graphicx}
\makeatletter
\newsavebox\pandoc@box
\newcommand*\pandocbounded[1]{% scales image to fit in text height/width
  \sbox\pandoc@box{#1}%
  \Gscale@div\@tempa{\textheight}{\dimexpr\ht\pandoc@box+\dp\pandoc@box\relax}%
  \Gscale@div\@tempb{\linewidth}{\wd\pandoc@box}%
  \ifdim\@tempb\p@<\@tempa\p@\let\@tempa\@tempb\fi% select the smaller of both
  \ifdim\@tempa\p@<\p@\scalebox{\@tempa}{\usebox\pandoc@box}%
  \else\usebox{\pandoc@box}%
  \fi%
}
% Set default figure placement to htbp
\def\fps@figure{htbp}
\makeatother

% load packages
\usepackage{geometry}
\usepackage{xcolor}
\usepackage{eso-pic}
\usepackage{fancyhdr}
\usepackage{sectsty}
\usepackage{fontspec}
\usepackage{titlesec}


%% Set page size with a wider right margin
\geometry{a4paper, total={170mm,257mm}, left=20mm, top=20mm, bottom=20mm, right=50mm}

%% Let's define some colours
\definecolor{light}{HTML}{E6E6FA}
\definecolor{highlight}{HTML}{800080}
\definecolor{dark}{HTML}{8200d7}

%% Let's add the border on the right hand side 
\AddToShipoutPicture{% 
    \AtPageLowerLeft{% 
        \put(\LenToUnit{\dimexpr\paperwidth-3cm},0){% 
            \color{dark}\rule{3cm}{\LenToUnit\paperheight}%
          }%
     }%
     % logo
    \AtPageLowerLeft{% start the bar at the bottom right of the page
        \put(\LenToUnit{\dimexpr\paperwidth-2.75cm},27.2cm){% move it to the top right
            \color{light}\includegraphics[width=2.6cm]{_extensions/PrettyPDF/ReefCloud_logo_stacked_REV.png}
          }%
     }%
}

%% Style the page number
\fancypagestyle{mystyle}{
  \fancyhf{}
  \renewcommand\headrulewidth{0pt}
  \fancyfoot[R]{\thepage}
  \fancyfootoffset{3.5cm}
}
\setlength{\footskip}{20pt}

%% style the chapter/section fonts
\chapterfont{\color{dark}\fontsize{20}{16.8}\selectfont}
\sectionfont{\color{dark}\fontsize{20}{16.8}\selectfont}
\subsectionfont{\color{dark}\fontsize{14}{16.8}\selectfont}
\titleformat{\subsection}
  {\sffamily\Large\bfseries}{\thesection}{1em}{}[{\titlerule[0.8pt]}]
  
% left align title
\makeatletter
\renewcommand{\maketitle}{\bgroup\setlength{\parindent}{0pt}
\begin{flushleft}
  {\sffamily\huge\textbf{\MakeUppercase{\@title}}} \vspace{0.3cm} \newline
  {\Large {\@subtitle}} \newline
  {\@author} \newline
  \@date
\end{flushleft}\egroup
}
\makeatother





%% Use some custom fonts
% \setsansfont{Ubuntu}[
    % Path=_extensions/nrennie/PrettyPDF/Ubuntu/,
    % Scale=0.9,
    % Extension = .ttf,
    % UprightFont=*-Regular,
    % BoldFont=*-Bold,
    % ItalicFont=*-Italic
    % ]

% \setmainfont{Ubuntu}[
    % Path=_extensions/nrennie/PrettyPDF/Ubuntu/,
    % Scale=0.9,
    % Extension = .ttf,
    % UprightFont=*-Regular,
    % BoldFont=*-Bold,
    % ItalicFont=*-Italic
    % ]
\usepackage{mathtools}
\KOMAoption{captions}{tableheading}
\makeatletter
\@ifpackageloaded{caption}{}{\usepackage{caption}}
\AtBeginDocument{%
\ifdefined\contentsname
  \renewcommand*\contentsname{Table of contents}
\else
  \newcommand\contentsname{Table of contents}
\fi
\ifdefined\listfigurename
  \renewcommand*\listfigurename{List of Figures}
\else
  \newcommand\listfigurename{List of Figures}
\fi
\ifdefined\listtablename
  \renewcommand*\listtablename{List of Tables}
\else
  \newcommand\listtablename{List of Tables}
\fi
\ifdefined\figurename
  \renewcommand*\figurename{Figure}
\else
  \newcommand\figurename{Figure}
\fi
\ifdefined\tablename
  \renewcommand*\tablename{Table}
\else
  \newcommand\tablename{Table}
\fi
}
\@ifpackageloaded{float}{}{\usepackage{float}}
\floatstyle{ruled}
\@ifundefined{c@chapter}{\newfloat{codelisting}{h}{lop}}{\newfloat{codelisting}{h}{lop}[chapter]}
\floatname{codelisting}{Listing}
\newcommand*\listoflistings{\listof{codelisting}{List of Listings}}
\makeatother
\makeatletter
\makeatother
\makeatletter
\@ifpackageloaded{caption}{}{\usepackage{caption}}
\@ifpackageloaded{subcaption}{}{\usepackage{subcaption}}
\makeatother
\makeatletter
\@ifpackageloaded{tcolorbox}{}{\usepackage[skins,breakable]{tcolorbox}}
\makeatother
\makeatletter
\@ifundefined{shadecolor}{\definecolor{shadecolor}{rgb}{.97, .97, .97}}{}
\makeatother
\makeatletter
\@ifundefined{codebgcolor}{\definecolor{codebgcolor}{named}{light}}{}
\makeatother
\makeatletter
\ifdefined\Shaded\renewenvironment{Shaded}{\begin{tcolorbox}[breakable, sharp corners, colback={codebgcolor}, boxrule=0pt, enhanced, frame hidden]}{\end{tcolorbox}}\fi
\makeatother

\usepackage{bookmark}

\IfFileExists{xurl.sty}{\usepackage{xurl}}{} % add URL line breaks if available
\urlstyle{same} % disable monospaced font for URLs
\hypersetup{
  pdftitle={Report},
  colorlinks=true,
  linkcolor={highlight},
  filecolor={Maroon},
  citecolor={Blue},
  urlcolor={highlight},
  pdfcreator={LaTeX via pandoc}}


\title{Report}
\usepackage{etoolbox}
\makeatletter
\providecommand{\subtitle}[1]{% add subtitle to \maketitle
  \apptocmd{\@title}{\par {\large #1 \par}}{}{}
}
\makeatother
\subtitle{Coral reefs habitat status and trends}
\author{Manuel Gonzalez-Rivero\textsuperscript{}}
\date{2025-03-02}

\begin{document}
\maketitle

\pagestyle{mystyle}


\section{Disclaimer}\label{disclaimer}

This report has been automatically generated with results from
monitoring that are publicly available on ReefCloud.ai. Neither the
authors or the Australian Institute of Marine Science provide any
guarantees or take responsibility for the content or interpretation of
these results. Users are advised to verify the data and consult with
experts for accurate interpretation and application of the information
presented in this report.

\newpage

\section{Introduction}\label{introduction}

This report provides a summary of the results from monitoring coral reef
habitats in this region using ReefCloud.ai. ReefCloud.ai is an advanced
platform designed to facilitate the collection, analysis, and
visualization of coral reef monitoring data. By leveraging cutting-edge
technology and data integration, ReefCloud.ai enables researchers and
stakeholders to assess the status and trends of coral reef ecosystems
effectively.

Coral reefs are among the most diverse and valuable ecosystems on Earth,
providing critical habitat for countless marine species, supporting
fisheries, protecting coastlines from erosion, and contributing to local
economies through tourism and recreation (Hoegh-Guldberg et al., 2007).
Despite their importance, coral reefs are facing unprecedented threats
from climate change, overfishing, pollution, and other human activities.
Recent studies have documented significant declines in coral cover and
reef health globally, with some regions experiencing losses of up to
50\% in the past few decades (Hughes et al., 2017; Bruno \& Selig,
2007).

Monitoring coral reefs is essential for understanding the extent and
drivers of these declines and for informing effective conservation and
management actions. Regular monitoring allows scientists to detect
changes in reef condition, identify areas of resilience, and evaluate
the effectiveness of conservation interventions (McClanahan et al.,
2012). However, traditional monitoring methods can be labor-intensive,
time-consuming, and costly, limiting their scalability and frequency.

ReefCloud.ai addresses these challenges by providing a user-friendly
platform that streamlines the process of data collection, analysis, and
reporting. The platform integrates data from various sources, including
underwater photographs, satellite imagery, and environmental sensors, to
provide a comprehensive view of reef health. Advanced analytical tools
and machine learning algorithms enable the rapid processing and
interpretation of large datasets, making it easier for researchers to
track changes over time and identify emerging threats.

In addition to its technical capabilities, ReefCloud.ai fosters
collaboration and data sharing among researchers, conservation
practitioners, and policymakers. By providing a centralized repository
for reef monitoring data, the platform promotes transparency and
facilitates the dissemination of findings to a broader audience. This
collaborative approach enhances the collective understanding of coral
reef dynamics and supports the development of coordinated conservation
strategies.

In summary, this report leverages the capabilities of ReefCloud.ai to
provide an in-depth assessment of coral reef habitats in the region. The
findings presented here will contribute to ongoing efforts to protect
and restore these vital ecosystems for future generations.

References: - Hoegh-Guldberg, O., et al.~(2007). Coral reefs under rapid
climate change and ocean acidification. Science, 318(5857), 1737-1742. -
Hughes, T. P., et al.~(2017). Global warming and recurrent mass
bleaching of corals. Nature, 543(7645), 373-377. - Bruno, J. F., \&
Selig, E. R. (2007). Regional decline of coral cover in the
Indo-Pacific: Timing, extent, and subregional comparisons. PLoS ONE,
2(8), e711. - McClanahan, T. R., et al.~(2012). Critical thresholds and
tangible targets for ecosystem-based management of coral reef fisheries.
Proceedings of the National Academy of Sciences, 109(41), 16282-16287.

\section{Methodology}\label{methodology}

\subsection{Study Area}\label{study-area}

The surveyed area, identified within the administrative boundaries of
Tỉnh Khánh Hòa in Vietnam, encompasses the following details:

\begin{itemize}
\tightlist
\item
  \textbf{Number of Sites}: 90
\item
  \textbf{Number of Photo Quadrats}: 11553
\item
  \textbf{Data Contributors}: Institute of Oceanography, Vietnam, Samara
  State University of Social Sciences and Education
\item
  \textbf{Source}: www.reefcloud.ai
\end{itemize}

The survey sites are distributed across the region
(Figure~\ref{fig-map}), detailed information provided in
Table~\ref{tbl-surveyed_sites}.

\begin{figure}[H]

\includegraphics[width=5in,height=\textheight,keepaspectratio]{figures/Site_Map.png}

\caption{\label{fig-map}Map of survey sites in the region.}

\end{figure}%

\subsection{Reef Habitats Status}\label{reef-habitats-status}

\subsubsection{Major Benthic cover}\label{major-benthic-cover}

Hard coral cover is a key indicator of reef health, representing the
proportion of the reef surface covered by live hard corals
(Hexacorallia). This measure is widely accepted and used globally to
assess the condition of coral reef habitats (Hughes et al., 2010). The
data in this report summarizes integrated and standardized percent cover
information from annotated photo-quadrats from multiple monitoring
projects in ReefCloud for the selected region, made publicly available
by project administrators on ReefCloud.ai.

ReefCloud analyzes monitoring images and automatically extracts hard
coral cover estimates from 50 points per image using Artificial
Intelligence (sensu Gonzalez-Rivero et al., 2014). Specifically,
ReefCloud employs Deep Learning with a Convolutional Neural Network
model pre-trained with monitoring data from the Australian Institute of
Marine Science. This model reduces the complexity of data describing
benthic composition from images. Through Transfer Learning, the
pre-trained model is adapted to specific locations and training data for
projects in the selected region, classifying 50 points on each survey
image to estimate the percent cover of hard coral and macroalgae.

For this report, all publicly available data from ReefCloud for the
region is compiled into a single dataset, including information about
the sampling source, depth, location, and number of photos analyzed to
produce hard coral and macroalgae cover. These datasets are analyzed
using hierarchical Bayesian statistical models, which predict annual
cover (Hard Coral and Macroalgae) and the uncertainty of these
estimates. The posterior predictions from the model across the region
and for each site are aggregated to median and lower/upper credible
intervals within 95\% of the distribution quantiles (Gelman et al.,
2013).

ReefCloud uses a statistical model to generate site-specific cover
estimates. The model considers transects per site and categorical depth
strata (0-5m and \textgreater{} 5m depth) as replicates to produce cover
estimates per site and year using a Hierarchical Bayesian model
(Equation~\ref{eq-model}).

\begin{equation}\phantomsection\label{eq-model}{
\begin{multlined}
y_{ij} \sim{} \beta_{}Bin(\pi_{ij},n) \\
log\left(\frac{\pi_{ij}}{1-\pi_{ij}}\right) \sim{} \beta_0  + \beta_1year_2 + \beta_2depth_{ij} + \gamma_{site}
\\
\\
\\Where:
\\
\beta_0 \sim{} N (0,1) 
\\
\beta_1 \sim{} N (0,1) 
\\
\beta_2 \sim{} N (0,1) 
\\
\gamma_{site} \sim{} t_{}(3,0,1) 
\\
\phi_{} \sim{} \gamma_{}(1,0.1)
\\
\end{multlined}
}\end{equation}

\subsection{Environmental Pressures}\label{environmental-pressures}

\subsubsection{Thermal Stress}\label{thermal-stress}

First observed in the early 1980s, mass coral bleaching has become one
of the most visible and damaging impacts of climate change on marine
ecosystems (Glynn, 1984; Hoegh-Guldberg, 1999). When water temperatures
exceed the average maximum summer temperature for extended periods,
corals can become thermally stressed, leading to coral bleaching and
potentially death (Baker et al., 2008). Bleaching occurs when corals
lose their symbiotic algae (zooxanthellae), which provide them with
energy and their distinctive colors. Severe bleaching can lead to
increased disease and mortality in corals (Hoegh-Guldberg, 1999). Over
the past decade, coral bleaching events have become more frequent,
extensive, and intense (Hughes et al., 2018).

To monitor thermal stress, we use Degree Heating Weeks (DHW), calculated
from satellite measurements of Sea Surface Temperature (SST) provided by
NOAA Coral Reef Watch (Liu et al., 2014). DHW indicates the accumulated
heat stress over the past 12 weeks (\textasciitilde3 months) and is a
reliable predictor of coral bleaching. The units for DHW are ``degree
C-weeks,'' combining the intensity and duration of heat stress. Research
shows that when DHW reaches 4 degrees C-weeks or higher, widespread
bleaching is often observed. At DHW levels above 8, significant coral
mortality is typically evidenced (Liu et al., 2014).

DHW is calculated by accumulating temperature readings that are more
than one degree Celsius above the historical maximum monthly mean
temperature for a given location. For example, if the temperature is 2°C
above the summer maximum monthly mean for 4 weeks, the DHW value is 8
DHW (2°C x 4 weeks). The thermal stress is accumulated over a 12-week
sliding window (Eakin et al., 2010).

The data in this report was produced by ReefCloud.ai using NOAA's Coral
Reef Watch products to provide insights into thermal stress exposure for
the reported region. NOAA's Coral Reef Watch products are widely used
and the longest operating DHW product. ReefCloud uses NOAA's Annual
Maximum Monthly Degree Heating Week, measured by daily global 5km
satellite estimations of SST. Values are derived using the Version 3.1
daily global 5Km CoralTemp satellite SST data product.

This report uses the proportion of reef exposed to moderate and severe
levels of thermal stress from DHW. This proportion is calculated in
ReefCloud using a combination of NOAA Coral Reef Watch DHW rasters with
spatial information on the distribution of coral reefs in this region
from the Allen Coral Atlas and the Global Distribution of Coral Reefs
(WCWC-008) datasets. These datasets provide spatially explicit
information on the location and area covered by coral reefs. When
overlaid with the DHW, the ReefCloud data platform calculates the extent
of reef area exposed to moderate (4-8 DHW), severe (\textgreater8 DHW),
or no major thermal stress (\textless4 DHW).

The thermal stress data layer is updated annually in ReefCloud as the
data becomes available on the NOAA Coral Reef Watch data server.

References: - Glynn, P. W. (1984). Widespread coral mortality and the
1982--83 El Niño warming event. Environmental Conservation, 11(2),
133-146. - Hoegh-Guldberg, O. (1999). Climate change, coral bleaching
and the future of the world's coral reefs. Marine and Freshwater
Research, 50(8), 839-866. - Baker, A. C., Glynn, P. W., \& Riegl, B.
(2008). Climate change and coral reef bleaching: An ecological
assessment of long-term impacts, recovery trends and future outlook.
Estuarine, Coastal and Shelf Science, 80(4), 435-471. - Hughes, T. P.,
et al.~(2018). Global warming transforms coral reef assemblages. Nature,
556(7702), 492-496. - Liu, G., et al.~(2014). Reef-scale thermal stress
monitoring of coral ecosystems: New 5-km global products from NOAA Coral
Reef Watch. Remote Sensing, 6(11), 11579-11606. - Eakin, C. M., et
al.~(2010). Monitoring coral reefs from space. Oceanography, 23(4),
118-133.

\subsubsection{Tropical Cyclones}\label{tropical-cyclones}

A tropical cyclone is a rapidly rotating strom system with low
atmospheric pressure at its calm centre (i.e., eye), inward spiralling
rainbands, and strong winds that dorms in areas of sufficiently warmn
sea surface temperature in the world's tropical regions. In the southern
hemisphere, these tropical storms are called cyclones and rotate in a
clockwise directions. In the northen hemisphere, cyclones are called
hurricanes (western hemisphere) or typhoons (eastern hemisphere) and
rotate in an anti-clockwise direction. If sufficiently long lasting, the
extermee winds generated by these storms can build powerful waves wich
can severely damage coral reefs and shorelines. By modelling where
extreme waves could form during a given cyclone, its is possible t
predict a cyclone ``damage zone'' beyond which major damage to reefs
will not lokely occurr. This zone is defined as the area within which
the average height of the top one-third of the waves likely meets or
exceeds four (04) metres (Hs - significant wave height). Here, we call
this the 4MW (Hs \textgreater= 4m) zone. Field data from 8 past cyclones
in the Great Barrier Reef and Western Australia has shown such zones
perform well at captuiring severe damage - noting that because of reef
vulnerability is the response is highly variable at \textless1km scales
and some parts of the reefs which the damage zone will not be damaged.
Mapping the 4MW cyclone damaage zone helps reef manages to: 1) spatially
target management responses after major tropical cyclones to promote
recovery at severly damaged sites, and 2) identify spatial patterns in
historic cyclone exposure to explain habitat condition trajectories.

The data presented here for tropcial cyclones is souced from
ReefCloud.ai. ReefCloud ais to souorce the most accurate data available
for the regions impacted by cyclone every year. This means that the
platform collate data from various databases to provide the best
insights possible on the estimated damage zone from tropicla cyclones in
a given region. For cyclones within the Australian Exclusive Economic
Zone, ReefCloud source data provided by the
\href{http://www.bom.gov.au/cyclone/tropical-cyclone-knowledge-centre/databases/}{Australian
Bureau of Meteorology's cyclone database}. For other regions outside
Australia, ReefCloud uses data sourced from the
\href{https://www.ncdc.noaa.gov/ibtracs/}{International Best Archive for
Climate Stewardship}

Once formed, tropical cyclones are remarkably predictable and
well-organised strom systems, making it possible to reconstruct the
spatial distribution of winds and waves around the eye to build
preducted wind and wave fields from a shore list of input data that is
freely available in the meteorological databases abovementioned
(location of the ye, central and ambient air pressure, size of the eye,
size of cyclone and cyclone forward speed and direction). The 4MW damage
zone model (Puptinen et al 2016) was used to generated the damage zone.
This model uses thuis base data to reconstruct the spatial distribution
of wind speeds around the cyclone eye every hour along its track. It
then searches in each location across the studu area for locations where
wund soeeds were sufficiently high and long-lasting to build significant
wave heights equal or above four (04) meters. The model summarise the
data by recording the number of hours for which such conditions
persisted. The resultant 4MW cyclone damage zone predicts the locations
where sufficient wave ennergy could cause major damage to coral reefs.

\subsection{Temporal trends on hard coral
cover}\label{temporal-trends-on-hard-coral-cover}

\section{Results}\label{results}

\subsection{Status of coral reef habitats in Tỉnh Khánh
Hòa}\label{status-of-coral-reef-habitats-in-tux1ec9nh-khuxe1nh-huxf2a}

\begin{figure}[H]

\centering{

\includegraphics[width=1\linewidth,height=\textheight,keepaspectratio]{figures/SiteCondition_HARD CORAL.png}

}

\caption{\label{fig-waffle}Proportion of monitored reef sites within the
region withing condition classifications based on coral cover range: A =
\textgreater50\% cover, B = 30-50\% cover, C = 10-30\% cover, D=
\textless10 \% cover.}

\end{figure}%

In 2024, 92\% of a total of 90 monitored reef sites exhibited Hard Coral
within 0 - 10 \% (Figure~\ref{fig-waffle})

\subsection{Recent Environmental
Disturbances}\label{recent-environmental-disturbances}

\begin{figure}[H]

\centering{

\includegraphics[width=1\linewidth,height=\textheight,keepaspectratio]{figures/Disturbance_impact.png}

}

\caption{\label{fig-env_dist}Proportion of reef area within the region
affected by environmental pressures during the reporting year}

\end{figure}%

\subsection{Temporal trends}\label{temporal-trends}

\pagebreak

\section{Interpretation Notes}\label{interpretation-notes}

\pagebreak

\section{References}\label{references}

\begin{itemize}
\tightlist
\item
  Hoegh-Guldberg, O., et al.~(2007). Coral reefs under rapid climate
  change and ocean acidification. Science, 318(5857), 1737-1742.
\item
  Hughes, T. P., et al.~(2017). Global warming and recurrent mass
  bleaching of corals. Nature, 543(7645), 373-377.
\item
  Bruno, J. F., \& Selig, E. R. (2007). Regional decline of coral cover
  in the Indo-Pacific: Timing, extent, and subregional comparisons. PLoS
  ONE, 2(8), e711.
\item
  McClanahan, T. R., et al.~(2012). Critical thresholds and tangible
  targets for ecosystem-based management of coral reef fisheries.
  Proceedings of the National Academy of Sciences, 109(41), 16282-16287.
\item
  Glynn, P. W. (1984). Widespread coral mortality and the 1982--83 El
  Niño warming event. Environmental Conservation, 11(2), 133-146.
\item
  Hoegh-Guldberg, O. (1999). Climate change, coral bleaching and the
  future of the world's coral reefs. Marine and Freshwater Research,
  50(8), 839-866.
\item
  Baker, A. C., Glynn, P. W., \& Riegl, B. (2008). Climate change and
  coral reef bleaching: An ecological assessment of long-term impacts,
  recovery trends and future outlook. Estuarine, Coastal and Shelf
  Science, 80(4), 435-471.
\item
  Hughes, T. P., et al.~(2018). Global warming transforms coral reef
  assemblages. Nature, 556(7702), 492-496.
\item
  Liu, G., et al.~(2014). Reef-scale thermal stress monitoring of coral
  ecosystems: New 5-km global products from NOAA Coral Reef Watch.
  Remote Sensing, 6(11), 11579-11606.
\item
  Eakin, C. M., et al.~(2010). Monitoring coral reefs from space.
  Oceanography, 23(4), 118-133.
\end{itemize}

\pagebreak

\section{Suppementary Materials}\label{suppementary-materials}

\subsection{Site description}\label{site-description}

\begin{longtable}[]{@{}llll@{}}

\caption{\label{tbl-surveyed_sites}Monitoring site information}

\tabularnewline

\toprule\noalign{}
Name & Longitude & Latitude & Images \\
\midrule\noalign{}
\endhead
\bottomrule\noalign{}
\endlastfoot
Rom island & 109.3125 & 12.17065 & 80 \\
NW Mun island & 109.3000 & 12.17123 & 60 \\
SW Mun island & 109.2966 & 12.16783 & 80 \\
Bai Dua Mun island & 109.3077 & 12.16970 & 80 \\
Dam Bay & 109.2920 & 12.18781 & 330 \\
Hon Cau & 109.3680 & 12.28336 & 830 \\
Bai San & 109.3273 & 12.19101 & 686 \\
Hon Mieu North East & 109.2340 & 12.19350 & 736 \\
Hon Tam South West & 109.2474 & 12.17161 & 500 \\
Hon Vung & 109.3574 & 12.27284 & 706 \\
Bai Lan & 109.2937 & 12.18215 & 679 \\
Bai Bang & 109.3242 & 12.22084 & 491 \\
Eo Co & 109.2435 & 12.23156 & 566 \\
Bich Dam & 109.3191 & 12.18501 & 608 \\
Mun Island North S2 & 109.3086 & 12.16937 & 241 \\
Mun Island South S3 & 109.3029 & 12.16331 & 320 \\
Hon Tam South & 109.2411 & 12.17317 & 160 \\
Hon Mot & 109.2782 & 12.17540 & 160 \\
Hon Rua & 109.2423 & 12.29037 & 160 \\
NW Mun Island & 109.3000 & 12.17123 & 320 \\
SW Mun Island & 109.2966 & 12.16783 & 320 \\
Ran Trao & 109.2147 & 12.62694 & 320 \\
Hon Den - Van Phong & 109.3036 & 12.60031 & 320 \\
Bai Nom - Cam Ranh & 109.2486 & 11.83194 & 320 \\
Bai Can Thuy Trieu & 109.2401 & 12.08650 & 160 \\
Hon Mieu North East & 109.2340 & 12.19350 & 320 \\
Hon Tam South West & 109.2474 & 12.17161 & 320 \\
Cu Hin & 109.2179 & 12.13928 & 320 \\
Song Lo & 109.2094 & 12.16830 & 160 \\
Eo Co & 109.2435 & 12.23156 & 320 \\
Bai Bang & 109.3242 & 12.22084 & 320 \\
Hon Lao South East & 109.2169 & 12.35808 & 160 \\
Hon Lao South & 109.2133 & 12.35775 & 160 \\
Hon Theo & 109.3036 & 12.47607 & 80 \\
Hon Do & 109.3518 & 12.47875 & 160 \\

\end{longtable}

\pagebreak

\subsection{Writing Tips for a Technical Report on Coral Reef Monitoring
Results}\label{writing-tips-for-a-technical-report-on-coral-reef-monitoring-results}

When writing a technical report on coral reef monitoring results for a
non-technical but knowledgeable audience, it is essential to communicate
complex scientific information clearly and effectively. Here are some
tips to help you achieve this:

\begin{enumerate}
\def\labelenumi{\arabic{enumi}.}
\item
  \textbf{Know Your Audience}: Understand the background and interests
  of your audience. Tailor your language and explanations to their level
  of knowledge, avoiding jargon and overly technical terms.
\item
  \textbf{Clear Structure}: Organize your report with a clear structure,
  including an introduction, methodology, results, discussion, and
  conclusion. Use headings and subheadings to guide the reader through
  the content.
\item
  \textbf{Visual Aids}: Use charts, graphs, maps, and images to
  illustrate key points and make data more accessible. Visual aids can
  help convey complex information quickly and effectively.
\item
  \textbf{Simplify Data Presentation}: Present data in a simplified
  manner. Use summary statistics, percentages, and averages to highlight
  key findings. Avoid overwhelming the reader with too much raw data.
\item
  \textbf{Contextualize Findings}: Provide context for your findings by
  comparing them to previous studies or historical data. Explain the
  significance of the results and their implications for coral reef
  conservation and management.
\item
  \textbf{Use Analogies and Examples}: Use analogies and real-world
  examples to explain complex concepts. This can help make the
  information more relatable and easier to understand.
\item
  \textbf{Highlight Key Messages}: Identify and emphasize the key
  messages you want your audience to take away from the report. Use
  bullet points, bold text, or call-out boxes to draw attention to these
  points.
\item
  \textbf{Provide Recommendations}: Offer clear and actionable
  recommendations based on your findings. Explain how these
  recommendations can be implemented and their potential impact on coral
  reef conservation.
\item
  \textbf{Cite Sources}: Include scientific references and citations to
  support your findings and provide credibility to your report. Use a
  consistent citation style throughout the document.
\item
  \textbf{Review and Edit}: Review your report for clarity, accuracy,
  and coherence. Seek feedback from colleagues or experts to ensure that
  your report is well-written and effectively communicates your
  findings.
\end{enumerate}

\subsubsection{Useful Writing Resources}\label{useful-writing-resources}

\begin{itemize}
\tightlist
\item
  \href{https://www.writing-skills.com/knowledge-hub/technical-writing/communicate-to-non-technical-audience}{Emphasis
  - Technical Writing}
\item
  \href{https://owl.purdue.edu/owl/subject_specific_writing/professional_technical_writing/index.html}{Purdue
  Online Writing Lab (OWL)}
\item
  \href{https://www.epa.gov/sites/default/files/2015-06/documents/technical-writing.pdf}{Technical
  Writing for Environmental Professionals}
\item
  \href{https://www.nature.com/scitable/ebooks/effective-communication-13953993/}{Effective
  Communication for Scientists}
\end{itemize}




\end{document}
